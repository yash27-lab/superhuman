\documentclass{article}
\usepackage{geometry}
\usepackage{graphicx, url, hyperref, amsmath, amssymb, mathtools, comment, xcolor}
\usepackage{amsthm}
\usepackage[most]{tcolorbox}
\tcbuselibrary{skins, breakable}
\usepackage[shortlabels]{enumitem}
\usepackage{mathscinet} % for \Dbar

\usepackage{booktabs}
\usepackage{tabularx}

% Geometry settings
\geometry{margin=1.5in}

% --- Theorems and Styles ---
\theoremstyle{plain}
\newtheorem{theorem}{Theorem}
\newtheorem{proposition}{Proposition}
\newtheorem{lemma}{Lemma}
\newtheorem{claim}{Claim}

\theoremstyle{definition}
\newtheorem*{example}{Example}
\newtheorem{remark}{Remark}[section]% Theorem environments

\theoremstyle{definition}
\newtheorem{definition}{Definition}
% --- Custom Commands ---
\newcommand{\Aletheia}{\emph{Aletheia}}
\renewcommand{\comment}[1]{\marginpar{{\tiny{#1}\normalfont\par}}}
\newcommand{\tony}[1]{{\color{red}Tony: #1}}

% --- Defined Colors ---
\definecolor{boxblue}{RGB}{0, 0, 150}
\definecolor{boxback}{RGB}{245, 245, 255}

% --- The Problem Environment ---
\newtcolorbox{problem}[1]{%
    colback=boxback,
    colframe=boxblue,
    fonttitle=\bfseries\large,
    title={#1},
    sharp corners,
    enhanced jigsaw, % Better frame handling for page breaks than just 'enhanced'
    breakable,       % <--- Allows the box to split across pages
    attach boxed title to top left={yshift=-2mm, xshift=2mm},
    boxed title style={colframe=boxblue, colback=boxblue},
    before skip=15pt plus 2pt,
    after skip=15pt plus 2pt,
    top=10pt, bottom=10pt, left=10pt, right=10pt
}

\newtcolorbox{solution}[1]{%
    colback=white,
    colframe=boxblue,
    fonttitle=\bfseries\large,
    title={#1},
    sharp corners,
    enhanced jigsaw, % Better frame handling for page breaks than just 'enhanced'
    breakable,       % <--- Allows the box to split across pages
    attach boxed title to top left={yshift=-2mm, xshift=2mm},
    boxed title style={colframe=boxblue, colback=boxblue},
    before skip=15pt plus 2pt,
    after skip=15pt plus 2pt,
    top=10pt, bottom=10pt, left=10pt, right=10pt
}



% Define custom commands for readability
\newcommand{\Q}{\mathbb{Q}}
\newcommand{\VV}{\mathbb{V}}
\newcommand{\frR}{\mathfrak{R}}
\newcommand{\co}{\colon}
\DeclareMathOperator{\Gr}{Gr}
\DeclareMathOperator{\aug}{aug}



\date{}


\begin{document}
\emph{The prompts and responses below have been edited only for
formatting compatibility.}


\begin{problem}{User Prompt}
Is it true that if $a_1<a_2<\cdots$ is a sequence of integers with\[\liminf a_n^{1/2^n}>1\]then\[\sum_{n=1}^\infty \frac{1}{a_na_{n+1}}\]is irrational?
\end{problem}


\begin{solution}{Model Response (\emph{Aletheia}, ablation run)}
    \textbf{Introduction.} In this paper, we examine the arithmetic nature of the series
\[ S = \sum_{n=1}^\infty \frac{1}{a_n a_{n+1}}, \]
where $(a_n)_{n \ge 1}$ is a strictly increasing sequence of integers. We assume the sequence exhibits double exponential growth, quantified by the condition:
\begin{equation} \label{eq:hypothesis}
L := \liminf_{n\to\infty} a_n^{1/2^n} > 1.
\end{equation}
This problem relates to classical results by Erd\H{o}s, Graham, and Badea concerning the irrationality of series with rapidly growing denominators \cite{Erdos1980, Badea1987}. We prove that condition \eqref{eq:hypothesis} is sufficient to guarantee the irrationality of $S$.

\section{Main Result}

\begin{theorem}
Let $(a_n)_{n \ge 1}$ be a strictly increasing sequence of integers satisfying \eqref{eq:hypothesis}. Then the sum $S$ is an irrational number.
\end{theorem}

\begin{proof}
We proceed by contradiction. Assume that $S$ is a rational number.

\paragraph{Reduction to a Tail Sequence.}
Since $L > 1$, the terms $a_n$ tend to infinity. Let $c$ be a constant such that $1 < c < L$. By the definition of the limit inferior, there exists an index $N_0$ such that for all $n \ge N_0$, $a_n^{1/2^n} > c$. Consequently, $a_n \ge 2$ for all $n \ge N_0$. The original series $S$ can be decomposed into a finite sum and an infinite tail:
\[ S = \sum_{n=1}^{N_0-1} \frac{1}{a_n a_{n+1}} + \sum_{n=N_0}^\infty \frac{1}{a_n a_{n+1}}. \]
Since the finite sum is rational, $S$ is rational if and only if the tail sum is rational. We define a shifted sequence $(b_n)_{n \ge 1}$ by $b_n = a_{N_0 + n - 1}$. The sequence $(b_n)$ satisfies:
\begin{enumerate}
\item $b_n \in \mathbb{Z}$, $b_n \ge 2$, and $b_{n+1} > b_n$ for all $n \ge 1$.
\item The sum $S' = \sum_{n=1}^\infty \frac{1}{b_n b_{n+1}}$ is rational. Let $S' = p/q$ for positive integers $p, q$.
\item The growth rate is preserved in the limit. Let $\mathcal{L} = \liminf_{n \to \infty} b_n^{1/2^n}$. We have:
\[ \mathcal{L} = \liminf_{n \to \infty} \left( a_{N_0+n-1}^{1/2^{N_0+n-1}} \right)^{2^{N_0-1}} = L^{2^{N_0-1}}. \]
Since $L > 1$ and $N_0 \ge 1$, we have $\mathcal{L} > 1$.
\end{enumerate}

\paragraph{Integer Constraints from Rationality.}
Let $P_n = \prod_{k=1}^n b_k$. Define the partial sum $S'_n = \sum_{k=1}^{n-1} \frac{1}{b_k b_{k+1}}$ and the remainder $R_n = S' - S'_n$. Substituting $S' = p/q$:
\[ R_n = \frac{p}{q} - S'_n \implies q P_n R_n = p P_n - q (P_n S'_n). \]
The term $P_n S'_n = \sum_{k=1}^{n-1} \frac{P_n}{b_k b_{k+1}}$ is an integer because for every $k < n$, the distinct factors $b_k$ and $b_{k+1}$ divide $P_n$. Thus, the quantity $K_n := q P_n R_n$ is an integer. Since the terms of the series are strictly positive, $R_n > 0$, implying $K_n \ge 1$. This yields a lower bound for the remainder:
\begin{equation} \label{eq:Rn_lower}
R_n \ge \frac{1}{q P_n}.
\end{equation}

\paragraph{Recurrence Bound on Sequence Growth.}
We establish an upper bound for $b_{n+1}$ derived from the properties of $R_n$. Using the telescoping inequality $\frac{1}{b_k b_{k+1}} < \frac{1}{b_k} - \frac{1}{b_{k+1}}$, we sum from $k=n$ to infinity:
\[ R_n = \sum_{k=n}^\infty \frac{1}{b_k b_{k+1}} < \frac{1}{b_n}. \]
From $K_n = q P_n R_n$, we have:
\[ K_n < \frac{q P_n}{b_n} = q P_{n-1} \quad (\text{with } P_0=1). \]
Applying this to index $n+1$, we obtain $K_{n+1} < q P_n$. Next, we use the recurrence relation for the remainder:
\[ R_n = \frac{1}{b_n b_{n+1}} + R_{n+1}. \]
Multiplying by $q P_{n+1} = q P_n b_{n+1}$:
\[ q P_{n+1} R_n = \frac{q P_n b_{n+1}}{b_n b_{n+1}} + q P_{n+1} R_{n+1}. \]
Simplifying the first term on the right-hand side using $P_n/b_n = P_{n-1}$:
\[ b_{n+1} (q P_n R_n) = q P_{n-1} + (q P_{n+1} R_{n+1}). \]
Substituting $K_n$ and $K_{n+1}$:
\[ b_{n+1} K_n = q P_{n-1} + K_{n+1}. \]
Since $K_n \ge 1$, we have $b_{n+1} \le b_{n+1} K_n$. Therefore:
\[ b_{n+1} \le q P_{n-1} + K_{n+1}. \]
Using the bound $K_{n+1} < q P_n$:
\[ b_{n+1} < q P_{n-1} + q P_n = q P_{n-1} (1 + b_n). \]
Since $b_n \ge 2$, $1+b_n \le 2 b_n$. Thus:
\[ b_{n+1} < 2 q P_{n-1} b_n = 2 q P_n. \]
Substituting this into $P_{n+1} = P_n b_{n+1}$, we obtain the growth constraint:
\begin{equation} \label{eq:Pn_recurrence}
P_{n+1} < 2 q P_n^2.
\end{equation}

\paragraph{Convergence of Sequence Limits.}
We analyze the asymptotic behavior of $P_n$. Taking the natural logarithm of \eqref{eq:Pn_recurrence}:
\[ \ln P_{n+1} < 2 \ln P_n + \ln(2q). \]
Dividing by $2^{n+1}$:
\[ \frac{\ln P_{n+1}}{2^{n+1}} < \frac{\ln P_n}{2^n} + \frac{\ln(2q)}{2^{n+1}}. \]
Let $y_n = 2^{-n} \ln P_n$. Then $y_{n+1} < y_n + 2^{-(n+1)} \ln(2q)$. Consider the auxiliary sequence $z_n = y_n + 2^{-n} \ln(2q)$. We observe that $z_n$ is strictly decreasing. To prove convergence, we show it is bounded below. Since $\mathcal{L} > 1$, for sufficiently large $n$, $b_n > c^{2^n}$ for some $c > 1$. This implies $P_n$ grows at least double exponentially, so $y_n$ is bounded away from 0. Thus $z_n$ converges, implying $y_n$ converges. Let $Y = \lim_{n\to\infty} y_n$ and define:
\[ \Pi = \lim_{n\to\infty} P_n^{1/2^n} = e^Y. \]
We determine the limit of the normalized sequence terms. Since $b_n = P_n / P_{n-1}$:
\[ b_n^{1/2^n} = \frac{P_n^{1/2^n}}{ (P_{n-1}^{1/2^{n-1}})^{1/2} }. \]
Taking limits as $n \to \infty$:
\[ \lim_{n\to\infty} b_n^{1/2^n} = \frac{\Pi}{\sqrt{\Pi}} = \sqrt{\Pi}. \]
Since the limit exists, it must equal the limit inferior $\mathcal{L}$. Therefore:
\begin{equation} \label{eq:Pi_L2}
\Pi = \mathcal{L}^2.
\end{equation}

\subsection*{Contradiction via Remainder Estimation}

We now derive a conflicting lower bound for $\Pi$. Fix any constant $C$ such that $1 < C < \mathcal{L}$. For sufficiently large $k$, $b_k > C^{2^k}$. The term $b_k b_{k+1}$ grows as $C^{2^k} C^{2^{k+1}} = C^{3 \cdot 2^k}$. The remainder $R_n$ is bounded by:
\[ R_n = \sum_{k=n}^\infty \frac{1}{b_k b_{k+1}} < \sum_{k=n}^\infty C^{-3 \cdot 2^k}. \]
For $C > 1$, this series is dominated by its first term (bounded by a geometric series with ratio $\le 1/2$ for large $n$):
\[ R_n < 2 C^{-3 \cdot 2^n}. \]
Recall the rationality lower bound \eqref{eq:Rn_lower}, $R_n \ge \frac{1}{q P_n}$. Combining these:
\[ \frac{1}{q P_n} < 2 C^{-3 \cdot 2^n} \implies P_n > \frac{1}{2q} C^{3 \cdot 2^n}. \]
Raising to the power $1/2^n$ and taking the limit $n \to \infty$:
\[ \Pi \ge \lim_{n\to\infty} (2q)^{-1/2^n} C^3 = C^3. \]
Since this holds for all $C < \mathcal{L}$, we conclude:
\begin{equation} \label{eq:Pi_L3}
\Pi \ge \mathcal{L}^3.
\end{equation}

\subsection*{Conclusion}

We have derived two relations: $\Pi = \mathcal{L}^2$ and $\Pi \ge \mathcal{L}^3$. Combining them yields $\mathcal{L}^2 \ge \mathcal{L}^3$. Since $\mathcal{L} > 1$, we divide by $\mathcal{L}^2$ to obtain:
\[ 1 \ge \mathcal{L}. \]
This contradicts the condition $\mathcal{L} > 1$. Therefore, the assumption that $S$ is rational implies a contradiction.
\end{proof}

\begin{thebibliography}{9}
\bibitem{Erdos1980} P. Erd\H{o}s and R. L. Graham, \textit{Old and New Problems and Results in Combinatorial Number Theory}, Monographies de L'Enseignement Math\'ematique, No. 28, Universit\'e de Gen\`eve, Geneva, 1980.
\bibitem{Badea1987} C. Badea, ``The irrationality of certain infinite series,'' \textit{Glasgow Mathematical Journal}, vol. 29, no. 2, pp. 221--228, 1987.
\end{thebibliography}
\end{solution}


\begin{solution}{Model Response (\emph{Aletheia}, initial run)}

\begin{theorem}
Let $(a_n)_{n \ge 1}$ be a strictly increasing sequence of integers such that
\[ \liminf_{n \to \infty} a_n^{1/2^n} > 1. \]
Then the infinite sum
\[ S = \sum_{n=1}^\infty \frac{1}{a_n a_{n+1}} \]
is irrational.
\end{theorem}

\begin{proof}
\textbf{1. Growth Estimates Based on Hypothesis} \\
Let $\lambda = \liminf_{n \to \infty} a_n^{1/2^n}$. By hypothesis, $\lambda > 1$. Select a constant $c$ such that $1 < c < \lambda$. By the definition of limit inferior, there exists an integer $N_0$ such that for all $n \ge N_0$:
\[ a_n^{1/2^n} > c \implies a_n > c^{2^n}. \]
Let $u_n = \ln a_n$. Taking the logarithm, we have $u_n > 2^n \ln c$.
Let $T_n = \sum_{i=1}^n u_i$. Bounding the sum from below:
\[ T_n > \sum_{i=N_0}^n 2^i \ln c = \ln c (2^{n+1} - 2^{N_0}). \]
Let $\alpha = \ln c > 0$. For sufficiently large $n$, we have:
\begin{equation}
T_n > \alpha 2^n.
\end{equation}
Since $a_n$ is a strictly increasing sequence of integers, $a_n \ge 1$ for all sufficiently large $n$. We assume $n \ge N_0$ throughout the remaining argument.

\vspace{1em}
\textbf{2. Assumption of Rationality and Lower Bound} \\
Assume, for the sake of contradiction, that $S$ is rational. Let $S = \frac{p}{q}$ for positive integers $p, q$.
Define the $n$-th partial sum $S_n$ and the remainder $r_n$:
\[ S_n = \sum_{k=1}^n \frac{1}{a_k a_{k+1}}, \qquad r_n = S - S_n = \sum_{k=n+1}^\infty \frac{1}{a_k a_{k+1}}. \]
Since the terms are positive, $S_n < S$, so $r_n > 0$.
Consider the denominators in $S_n$. Each term is of the form $\frac{1}{a_k a_{k+1}}$. Notice that $a_k a_{k+1}$ divides the product $P_{n+1} = \prod_{i=1}^{n+1} a_i$.
Let $L_n = \operatorname{lcm}(a_1 a_2, \dots, a_n a_{n+1})$. Then $L_n \le P_{n+1}$.
Writing $S_n = \frac{M_n}{L_n}$ for some integer $M_n$, we have:
\[ r_n = \frac{p}{q} - \frac{M_n}{L_n} = \frac{p L_n - q M_n}{q L_n}. \]
Since $r_n > 0$, the numerator is a positive integer, so $p L_n - q M_n \ge 1$.
Thus:
\begin{equation}
r_n \ge \frac{1}{q L_n} \ge \frac{1}{q P_{n+1}}.
\end{equation}
Using the relation $P_{n+1} = a_{n+1} P_n$, we write this as:
\[ r_n \ge \frac{1}{q a_{n+1} P_n}. \]

\vspace{1em}
\textbf{3. Universal Upper Bound for Remainder} \\
We now bound $r_{n+1} = \sum_{k=n+2}^\infty \frac{1}{a_k a_{k+1}}$.
Since $(a_k)$ is strictly increasing, $a_{k+1} > a_k$, so $\frac{1}{a_k a_{k+1}} < \frac{1}{a_k^2}$.
Thus, $r_{n+1} < \sum_{k=n+2}^\infty \frac{1}{a_k^2}$.

We partition the indices $k \ge n+2$ into sets $I_j$:
\[ I_j = \{ k \ge n+2 \mid 2^j a_{n+2} \le a_k < 2^{j+1} a_{n+2} \}, \quad j = 0, 1, 2, \dots \]
For any $k \in I_j$, we have $\frac{1}{a_k^2} \le \frac{1}{(2^j a_{n+2})^2}$. For any $k \in I_j$, $c^{2^k} < a_k < 2^{j+1} a_{n+2}$. Taking logs:
\[ 2^k \ln c < (j+1)\ln 2 + \ln a_{n+2} \implies k < \log_2 \left( \frac{(j+1)\ln 2 + \ln a_{n+2}}{\ln c} \right). \]
Let $L = \ln a_{n+2}$. There exists a constant $C > 0$ such that $|I_j| \le C (\ln L + j)$.
Substituting this into the sum:
\[ r_{n+1} < \sum_{j=0}^\infty \frac{C(\ln L + j)}{(2^j a_{n+2})^2} = \frac{C}{a_{n+2}^2} \left( \ln L \sum_{j=0}^\infty \frac{1}{4^j} + \sum_{j=0}^\infty \frac{j}{4^j} \right). \]
Thus, for some constant $K$:
\begin{equation}
r_{n+1} < \frac{K \ln \ln a_{n+2}}{a_{n+2}^2}.
\end{equation}

\vspace{1em}
\textbf{4. Deriving the Recurrence Relation} \\
Recall $r_n = \frac{1}{a_{n+1} a_{n+2}} + r_{n+1}$. Combining (2) and (3):
\[ \frac{1}{q a_{n+1} P_n} \le \frac{1}{a_{n+1} a_{n+2}} + \frac{K \ln \ln a_{n+2}}{a_{n+2}^2}. \]
Multiplying by $a_{n+1} a_{n+2}$ and using $a_{n+1}/a_{n+2} < 1$:
\[ \frac{a_{n+2}}{q P_n} < 1 + K \ln \ln a_{n+2}. \]
Taking logarithms ($u_n = \ln a_n$, $T_n = \ln P_n$):
\[ u_{n+2} < \ln q + T_n + \ln(1 + K \ln u_{n+2}). \]
For any $\epsilon > 0$ and sufficiently large $n$, $\ln(1 + K \ln u_{n+2}) < \epsilon u_{n+2}$. Thus:
\[ (1 - \epsilon) u_{n+2} < T_n + \ln q. \]
Substituting $u_{n+2} = T_{n+2} - T_{n+1}$:
\[ T_{n+2} < T_{n+1} + \frac{1}{1-\epsilon} T_n + C_\epsilon. \]

\vspace{1em}
\textbf{5. Contradiction} \\
The characteristic equation $x^2 - x - \frac{1}{1-\epsilon} = 0$ has a positive root:
\[ \rho(\epsilon) = \frac{1 + \sqrt{1 + \frac{4}{1-\epsilon}}}{2}. \]
For $\epsilon = 0.05$, $\rho(0.05) \approx 1.64 < 2$. Thus $T_n = O(1.7^n)$.
However, from Step 1, $T_n > \alpha 2^n$. This implies:
\[ \alpha 2^n < C (1.7)^n \implies \alpha < C \left( \frac{1.7}{2} \right)^n. \]
As $n \to \infty$, this implies $\alpha \le 0$, contradicting $\alpha > 0$.
The assumption that $S$ is rational must be false.
\end{proof}

\textbf{Conclusion:} The sum $S = \sum_{n=1}^\infty \frac{1}{a_n a_{n+1}}$ is irrational.

\end{solution}



\begin{problem}{User Prompt}
Find a meaningful, interesting generalization of this theorem or its proof.

\textbf{Theorem}
Fix $d\ge2$, and let $c_d$ be the unique positive real root  of $x^{d}-x^{d-1}-1=0$. Suppose a strictly increasing sequence $\{a_n\}$ of integers satisfy
$$ \liminf_{n \to \infty} a_n^{1/b^n} > 1$$
for some $b>c_d$. Then the infinite sum
$$ S = \sum_{n=1}^\infty \frac{1}{a_n a_{n+1}\cdots a_{n+d-1}} $$
is irrational.

\textbf{Proof}
For two functions $f$ and $g$, we write $f\precsim g$ when there exists some constant $C>0$ satisfying $f\le Cg$. Assume for contradiction that $S=p/q$.  We may assume there is some constant $c>1$ such that $a_n>c^{b^n}$ for all $n$. We set
$$P_n:=\prod_1^n a_i,\quad  S_{n} = \sum_{m=1}^n \frac{1}{a_m a_{m+1}\cdots a_{m+d-1}} .$$
Since $S>S_n$ for all $n$, we have
$$r_{n} := S - S_n=\frac{p}{q}-\frac{*}{P_{n+d-1}} \ge \frac1{qP_{n+d-1}}.$$
Let us define
$$ I_j(n):= \{ k \ge n+1 \mid 2^j a_{n+1}^{d} \le a_k^{d} < 2^{j+1} a_{n+1}^{d} \}, \quad j = 0, 1, 2, \dots $$
so that each $k \in I_j(n)$ satisfies
$$c^{d\cdot b^k}<a_k^d < 2^{j+1} a_{n+1}^d.$$
The cardinality of $I_j(n)$ can be bounded from above as
$$
|I_{j}(n)|\le \max_{k\in I_{j}(n)}k \le \log_{b}\left(\frac{\log_{c}(2^{j+1}a_{n+1}^d)}{d}\right)\precsim \log j+\log\log a_{n+1}.
$$
It follows that
$$ r_{n} < \sum_{k=n+1}^\infty \frac{1}{a_k^{d}} = \sum_{j=0}^\infty \sum_{k \in I_j(n)} \frac{1}{a_k^{d}}
\le
\sum_{j=0}^\infty  \frac{| I_j(n)|}{2^j a_{n+1}^{d}}
\precsim \sum_{j=0}^\infty \frac{\log j+\log\log a_{n+1} }{2^j a_{n+1}^d}\precsim \frac{ \log \log a_{n+1}}{a_{n+1}^{d}}.$$
Combining the lower and the upper bounds for $r_n$ and $r_{n+1}$, we have
\begin{align*}\frac{1}{q P_{n+d-1}} &\le r_{n}= \sum_{j=1}^{d-1}\frac{1}{a_{n+j} \cdots a_{j+n+d-1}} + r_{n+d-1}\\
&\precsim\frac{d-1}{a_{n+1}\cdots a_{n+d}} + \frac{\log\log a_{n+d}}{a_{n+d}^d}\le \frac{\log\log a_{n+d}}{a_{n+1}\cdots a_{n+d}}. \end{align*}
Fix an arbitrarily small $\epsilon>0$. Multiplying both sides by $P_{n+d}$, we obtain that
$$a_{n+d}=P_{n+d}/P_{n+d-1}\precsim P_{n}\log\log a_{n+d}
\precsim P_{n}a_{{n+d}}^\epsilon=P_{n}(P_{n+d}/P_{n+d-1})^\epsilon.$$

It follows that we have $P_{n+d}\precsim P_{n+d-1}P_{n}^{1/(1-\epsilon)}$. The sequence $u_n:=\log_{c} P_n$ satisfies the recursive inequality
$$u_{n+d}\le u_{n+d-1}+\frac1{1-\epsilon} u_{n} + K$$
for some constant $K=K(\epsilon)>0$. For a sufficiently small $\epsilon$, the equation $$x^d=x^{d-1}+\frac{1}{1-\epsilon}$$
has exactly one positive real root $s=s(d,\epsilon)$, slightly larger than $c_d$, and one or zero negative real root, depending on the parity of $d$; moreover, the negative real root, if exists, belong to $(-1,0)$. By choosing a sufficiently small $\epsilon$ satisfying $s=s(d,\epsilon)<b$,  we have that $u_n\precsim s^n$. This contradicts the estimate $\log_c P_n\ge \log_{c} a_{n}\ge b^n$. **QED**
\end{problem}

\begin{solution}{Model Response  (Gemini Deep Think)}
A meaningful and natural generalization of this theorem extends the result to sums of reciprocals of products with **arbitrary integer weights**.

The original theorem relies on a product of $d$ consecutive terms, effectively assigning a weight of $1$ to every integer in a "sliding window." By generalizing these exponents, we can quantify how the density or sparsity of the denominator affects the growth rate required for the series to be irrational.

\subsection*{Theorem (Generalized)}

Fix an integer $d \ge 1$ and a tuple of non-negative integers $\mathbf{w} = (w_0, w_1, \dots, w_{d-1})$ with $w_{d-1} \ge 1$. Let $W = \max(w_0, \dots, w_{d-1})$.

Let $c_{\mathbf{w}}$ be the unique real root greater than 1 of the polynomial:

$$ P(x) = (x-1)\sum_{j=0}^{d-1} w_j x^j - W x^{d-1} $$

Suppose a strictly increasing sequence of integers $\{a_n\}$ satisfies

$$ \liminf_{n \to \infty} a_n^{1/b^n} > 1 $$

for some $b > c_{\mathbf{w}}$. Then the infinite sum

$$ S = \sum_{n=1}^\infty \frac{1}{a_n^{w_0} a_{n+1}^{w_1} \cdots a_{n+d-1}^{w_{d-1}}} $$

is irrational.

---

\subsection*{Proof Sketch}

The generalization follows the "gap principle" strategy used in the original proof, balancing the decay of the tail against the growth of the common denominator.

1. \textbf{The Common Denominator (LCM):}

    Assume for contradiction that $S = p/q$. Let $S_n$ be the $n$-th partial sum. The denominators of the terms in $S_n$ involve integers up to $a_{n+d-1}$.

    Because the window shifts by 1 at each step, any specific integer $a_k$ appears in $d$ consecutive terms with exponents $w_{d-1}, \dots, w_0$. When taking the Least Common Multiple (LCM) to form the common denominator $Q_n$, the exponent of $a_k$ is bounded by the maximum weight $W$.

    $$ Q_n \lesssim \prod_{k=1}^{n+d-1} a_k^W \approx P_{n+d-1}^W $$

2. \textbf{The Tail Estimate:}

    The remainder $r_n = S - S_n$ is dominated by the first term of the tail (index $n+1$).

    $$ r_n \approx \frac{1}{a_{n+1}^{w_0} a_{n+2}^{w_1} \cdots a_{n+d}^{w_{d-1}}} $$

3. \textbf{The Inequality:}

    For $S$ to be rational, the error $r_n$ cannot be "too small" relative to the denominator $Q_n$. Specifically, $r_n \ge 1/(q Q_n)$. This implies $Q_n r_n \gtrsim 1$. Substituting the bounds:

    $$ P_{n+d-1}^W \cdot \frac{1}{\prod_{j=0}^{d-1} a_{n+1+j}^{w_j}} \gtrsim 1 \implies \prod_{j=0}^{d-1} a_{n+1+j}^{w_j} \lesssim P_{n+d-1}^W $$

    Taking logarithms (assuming $\log a_k \sim b^k$) and dividing by $b^{n+1}$:

    $$ \sum_{j=0}^{d-1} w_j b^j \lesssim W \frac{b^{d-1}}{b-1} $$

    Rearranging this inequality yields the polynomial $P(b) \le 0$. If $b > c_{\mathbf{w}}$ (the root), this inequality is violated for large $n$, proving irrationality.

---

\textbf{Why this is Meaningful}

1. \textbf{It Unifies "Dense" and "Sparse" Problems
}
\textbf{The Original Case:} If $\mathbf{w}=(1, \dots, 1)$, then $W=1$. The sum is $\frac{x^d-1}{x-1}$.

    $$ (x-1)\frac{x^d-1}{x-1} - 1\cdot x^{d-1} = x^d - x^{d-1} - 1 = 0 $$

    This recovers the original constant $c_d$.

\textbf{The "Gap" Case:} Consider $S = \sum \frac{1}{a_n a_{n+2}}$. Here $\mathbf{w}=(1, 0, 1)$ and $W=1$. The polynomial becomes:

    $$ (x-1)(1+x^2) - x^2 = x^3 - 2x^2 + x - 1 = 0 $$

    The root is $c \approx 1.75$, which is larger than the $c \approx 1.46$ required for the consecutive block $a_n a_{n+1} a_{n+2}$. This quantifies the intuition that **"missing terms" weaken the convergence**, forcing the sequence $a_n$ to grow faster to guarantee irrationality.

2. \textbf{It Handles Heavy Weights}

\textbf{Sylvester-Type Series:} If $d=1$ and $w_0=1$, the polynomial is $(x-1)(1) - 1 = x-2$, recovering the classic result that $\sum 1/a_n$ is irrational if $a_{n+1} \sim a_n^2$ (limit $b=2$).

If we increase the weight of the \textit{last} term (e.g., $1/(a_n a_{n+1}^2)$), the root increases, making irrationality harder to prove. This is because the "cost" of the LCM ($W=2$) rises, while the benefit to the tail decay is limited to the future terms.

3. \textbf{Universal Irrationality for Double-Exponential Sequences}

Since $c_d \to 1$ as $d \to \infty$ for the uniform case, this generalization implies a powerful corollary:

For \textbf{any} sequence $\{a_n\}$ with double-exponential growth (any $b > 1$), there exists a sufficiently large block size $d$ such that $\sum \frac{1}{a_n \cdots a_{n+d-1}}$ is irrational.\end{solution}
\end{document}
